\documentclass{anstrans}
%%%%%%%%%%%%%%%%%%%%%%%%%%%%%%%%%%%
\title{A nuclear energy-based approach for dealing with the duck curve}
\author{Roberto E. Fairhurst Agosta}

\institute{
University of Illinois at Urbana-Champaign, Dept. of Nuclear, Plasma, and Radiological Engineering\\
ref3@illinois.edu
}

%%%% packages and definitions (optional)
\usepackage{graphicx} % allows inclusion of graphics
\usepackage{booktabs} % nice rules (thick lines) for tables
\usepackage{microtype} % improves typography for PDF
\usepackage{xspace}
\usepackage{tabularx}
\usepackage{floatrow}
\usepackage{subcaption}
\usepackage{enumitem}
\usepackage{placeins}
\usepackage{amsmath}
\usepackage[acronym,toc]{glossaries}
\include{acros}
\makeglossaries

\usepackage[printwatermark]{xwatermark}
\usepackage{xcolor}
\usepackage{graphicx}
\usepackage{lipsum}
\usepackage{indentfirst}

\newcommand{\SN}{S$_N$}
\renewcommand{\vec}[1]{\bm{#1}} %vector is bold italic
\newcommand{\vd}{\bm{\cdot}} % slightly bold vector dot
\newcommand{\grad}{\vec{\nabla}} % gradient
\newcommand{\ud}{\mathop{}\!\mathrm{d}} % upright derivative symbol

\newcolumntype{c}{>{\hsize=.56\hsize}X}
\newcolumntype{b}{>{\hsize=.7\hsize}X}
\newcolumntype{s}{>{\hsize=.74\hsize}X}
\newcolumntype{f}{>{\hsize=.1\hsize}X}
\newcolumntype{a}{>{\hsize=.45\hsize}X}
%\usepackage[pagestyles]{titlesec}
%\titleformat*{\subsection}{\normalfont}
%\titleformat{\section}{\bfseries}{Item \thesection.\ }{0pt}{}

%\newwatermark[allpages,color=gray!50,angle=45,scale=3,xpos=0,ypos=0]{DRAFT}

\begin{document}
%%%%%%%%%%%%%%%%%%%%%%%%%%%%%%%%%%%%%%%%%%%%%%%%%%%%%%%%%%%%%%%%%%%%%%%%%%%%%%%%
\section{Introduction}
\label{intro}

Energy is one of the strongest contributors to economic growth.
In the future, economies will continue to grow, populations will do so too, and their energy demand will accompany such growth \cite{burke_impact_2018}.
Meeting these future needs requires the development of clean energy sources as environmental concerns continue to rise.
As seen in Figure \ref{fig:ghg}, electricity generation is one of the economic sectors that produced the most \glspl{GHG} in the \gls{US} in 2017.
As \gls{CO2} is the main component in \glspl{GHG}, decarbonizing electricity generation will allow us to meet the increases in energy demand and address the environmental concerns at the same time.

\begin{figure}[H]
	\centering
	\includegraphics[width=0.7\linewidth]{figures/total-ghg-2019-caption.jpg}
	\hfill
	\caption{Total U.S. GHG Emissions by Economic Sector in 2017 \cite{us_epa_sources_2020}.}
	\label{fig:ghg}
\end{figure}

% word on solar energy and the duck curve
To address these concerns, utilities rely more on intermittent renewable energy resources, wind and solar \cite{ming_resource_2019}.
However, high solar energy adoption creates a challenge. The need for electricity generators to quickly ramp up increases when the sun sets and the contribution from \gls{PV} falls \cite{us_department_of_energy_confronting_2017}.
The California ISO developed the “duck chart (or duck curve) depicted in Figure \ref{fig:duck} to
illustrate the difference between forecasted load and expected electricity production
from solar during a typical March day \cite{bouillon_prepared_2014}.

\begin{figure}[H]
	\centering
	\includegraphics[width=0.7\linewidth]{figures/caiso-duck.png}
	\hfill
	\caption{The duck curve \cite{bouillon_prepared_2014}.}
	\label{fig:duck}
\end{figure}

% solutions to the duck curve
The easiest solution to a demand ramp up is the increase of dispatchable generation, such as natural gas and coal \cite{bouillon_prepared_2014}.
Nonetheless, adding new gas and coal generation capacity is not consistent with reductions in carbon emissions.
Hence, our focus drifts to other potential low-carbon solutions, like nuclear generation and electricity storage \cite{ming_resource_2019}.

This article analyses the likelihood of the duck curve in the \gls{US} grid system. Then, it narrows down its focus on the \gls{UIUC}'s grid and studies some possible approaches to mitigate the duck curve effects. These approaches rely on nuclear energy, the production of hydrogen, and the use of hydrogen as an energy storage mechanism. 
% Link missing between this paragraph and the next section

%%%%%%%%%%%%%%%%%%%%%% TO DO
% Section \ref{section:hydroprod} briefly introduces two hydrogen production methods and their energy requirements. Section \ref{section:methodology}

\section{Nuclear Co-Generation}
\label{section:hydroprod}

This section presents the electrolysis process as a method to couple hydrogen production to a nuclear reactor.
Water electrolysis converts electric and thermal energy into chemical energy stored in hydrogen \cite{hi2h2_highly_2007}.
The process enthalpy change $\Delta H$ determines the required energy for the electrolysis reaction to take place.
Part of the energy corresponds to electric energy $\Delta G$ and the rest of it to thermal energy $T \cdot \Delta S$ \ref{fig:electro}.
This gives the relation
\begin{equation}
\Delta H = \Delta G + T \Delta S
\end{equation}
.
In liquid water electrolysis or \gls{LTE}, electricity provides the thermal energy.
In steam electrolysis or \gls{HTE}, a high temperature heat source is necessary to supply the thermal energy.

\begin{figure}[H]
	\centering
	\includegraphics[width=0.7\linewidth]{figures/ele-heat_curve.png}
	\hfill
	\caption{Energy consumption of an ideal electrolysis process \cite{hi2h2_highly_2007}.}
	\label{fig:electro}
\end{figure}

In this analysis, the energy source (both electric and thermal) is a nuclear reactor with co-generation capabilities.
The nuclear reactor produces electricity to supply the grid ($P_E$) and electricity ($P_{EH2}$)and thermal energy ($P_{TH2}$) to produce hydrogen \ref{fig:h2diag}.
$\beta$ and $\gamma$ determine the distribution of reactor thermal power $P_{th}$ into $P_E$, $P_{EH2}$, and $P_{TH2}$.
$\eta$ is the thermal-to-electric energy conversion ration.

\begin{figure}[H]
	\centering
	\includegraphics[width=0.7\linewidth]{figures/hte-figure0.png}
	\hfill
	\caption{.}
	\label{fig:h2diag}
\end{figure}

\section{Methodology}
\label{method}

We predict the total electricity generation in the \gls{US} for 2050 by doing a linear regression on the data of the total electricity generated in the US by year \cite{us_energy_information_administration_electric_2020}, Figure \ref{fig:us-pred1}.
Using the same method, we also estimate the solar electricity generation for 2050, Figure \ref{fig:us-pred2}.
The predictions allow us to determine a growth rate for the total ($GR_{T}$) and solar electricity generation ($GR_{S}$), eq. \ref{GR}. Where $T_i$ and $S_i$ correspond to the total and solar energy generation, respectively, in year $i$.

\begin{equation}
\begin{split}
GR_{T} &= \frac{T_{2050}} {T_{2019}}
\\
GR_{S} &= \frac{S_{2050}} {S_{2019}}
\end{split}
\label{GR}
\end{equation}

To determine the electricity generated by hour we need to choose a day first. We focus on the spring, when the solar energy produced is higher as it is sunny but the total demand it is low since people are not using electricity for air conditioning or heating \cite{us_department_of_energy_confronting_2017}.

Then, we apply the growth rate to the electricity generated data by hour and we obtain a prediction on the electricity generation profile by hour in 2050.
The hourly data is available at \cite{eia_united_2020}.
We repeat this process on \gls{UIUC} electricity generation by hour data.
We apply the growth rate calculated for the \gls{US} because the solar energy generation data is scarce.

Once that process is complete, we propose a nuclear reactor size ($P_e$) that we will work at full capacity.
Due to the duck curve, the electricity produced by the reactor at time $i$ ($P_i$) will exceed the demand at time $i$ (D$_i$), creating and excess of thermal energy at time $i$ ($E_i$) that will energize the hydrogen plant. We express this variables in terms of the variables in Figure \ref{fig:h2diag}, eq. \ref{eq:excess1}.

\begin{equation}
\begin{split}
P_i &= P_{th} \eta
\\
D_i &= P_{E} = \eta \beta P_{th}
\\
E_i &= \frac{P_i-D_i}{\eta}
\end{split}
\label{eq:excess1}
\end{equation}

\begin{equation}
E_i &= (1-\beta) P_{th}
\label{eq:excess2}
\end{equation}

Into Hydrogen:
Figure \ref{fig:h2diag}

\begin{equation}
\begin{split}
\Delta G = P_{EH2} &= \eta \gamma (1-\beta) P_{th}
\\
T\Delta S = P_{TH2} &= (1-\gamma) (1-\beta) P_{th}
\end{split}
\label{eq:hydro}
\end{equation}

% Word on the US yearly data
\begin{figure}[H]
	\centering
	\includegraphics[width=0.7\linewidth]{figures/us-prediction1.png}
	\hfill
	\caption{Prediction on the total electricity generation in the \gls{US} for 2050}
	\label{fig:us-pred1}
\end{figure}

\begin{figure}[H]
	\centering
	\includegraphics[width=0.7\linewidth]{figures/us-prediction2.png}
	\hfill
	\caption{Prediction on the solar electricity generation in the \gls{US} for 2050}
	\label{fig:us-pred2}
\end{figure}

\section{Results}

In spring of 2019, US solar generation peak on April 4 \fig{fig:us-duck}.
In spring of 2019, UIUC solar generation peak on April 4 \fig{fig:uiuc-duck}.

\begin{figure}[H]
	\centering
	\includegraphics[width=0.7\linewidth]{figures/duck-curve4.png}
	\hfill
	\caption{.}
	\label{fig:us-duck}
\end{figure}

\begin{figure}[H]
	\centering
	\includegraphics[width=0.7\linewidth]{figures/uiuc-duck.png}
	\hfill
	\caption{.}
	\label{fig:uiuc-duck}
\end{figure}

\begin{figure}[H]
	\centering
	\includegraphics[width=0.7\linewidth]{figures/uiuc-hydro2.png}
	\hfill
	\caption{.}
	\label{fig:uiuc-hydro2}
\end{figure}

\begin{figure}[H]
	\centering
	\includegraphics[width=0.7\linewidth]{figures/uiuc-hydro3.png}
	\hfill
	\caption{.}
	\label{fig:uiuc-hydro3}
\end{figure}

\section{Conclusion}



%%%%%%%%%%%%%%%%%%%%%%%%%%%%%%%%%%%%%%%%%%%%%%%%%%%%%%%%%%%%%%%%%%%%%%%%%%%%%%%%
\bibliographystyle{ans}
\bibliography{bibliography}
\end{document}
