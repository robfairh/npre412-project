\documentclass{anstrans}
%%%%%%%%%%%%%%%%%%%%%%%%%%%%%%%%%%%
\title{A nuclear energy-based approach for dealing with the duck curve}
\author{Roberto E. Fairhurst Agosta}

\institute{
University of Illinois at Urbana-Champaign, Dept. of Nuclear, Plasma, and Radiological Engineering\\
ref3@illinois.edu
}

%%%% packages and definitions (optional)
\usepackage{graphicx} % allows inclusion of graphics
\usepackage{booktabs} % nice rules (thick lines) for tables
\usepackage{microtype} % improves typography for PDF
\usepackage{xspace}
\usepackage{tabularx}
\usepackage{floatrow}
\usepackage{subcaption}
\usepackage{enumitem}
\usepackage{placeins}
\usepackage{amsmath}
\usepackage[acronym,toc]{glossaries}
\include{acros}
\makeglossaries

\usepackage[printwatermark]{xwatermark}
\usepackage{xcolor}
\usepackage{graphicx}
\usepackage{lipsum}
\usepackage{indentfirst}

\newcommand{\SN}{S$_N$}
\renewcommand{\vec}[1]{\bm{#1}} %vector is bold italic
\newcommand{\vd}{\bm{\cdot}} % slightly bold vector dot
\newcommand{\grad}{\vec{\nabla}} % gradient
\newcommand{\ud}{\mathop{}\!\mathrm{d}} % upright derivative symbol

\newcolumntype{c}{>{\hsize=.56\hsize}X}
\newcolumntype{b}{>{\hsize=.7\hsize}X}
\newcolumntype{s}{>{\hsize=.74\hsize}X}
\newcolumntype{f}{>{\hsize=.1\hsize}X}
\newcolumntype{a}{>{\hsize=.45\hsize}X}
%\usepackage[pagestyles]{titlesec}
%\titleformat*{\subsection}{\normalfont}
%\titleformat{\section}{\bfseries}{Item \thesection.\ }{0pt}{}

%\newwatermark[allpages,color=gray!50,angle=45,scale=3,xpos=0,ypos=0]{DRAFT}

\begin{document}
%%%%%%%%%%%%%%%%%%%%%%%%%%%%%%%%%%%%%%%%%%%%%%%%%%%%%%%%%%%%%%%%%%%%%%%%%%%%%%%%
\section{Introduction}
\label{intro}

Energy is one of the strongest contributors to economic growth.
In the future, economies will continue to grow, populations will do so too, and their energy demand will accompany such growth \cite{burke_impact_2018}.
Meeting these future needs requires the development of clean energy sources as environmental concerns continue to rise.
As seen in Figure \ref{fig:ghg}, electricity generation is one of the economic sectors that produced the most \glspl{GHG} in the \gls{US} in 2017.
As \gls{CO2} is the main component in \glspl{GHG}, decarbonizing electricity generation will allow us to meet the increases in energy demand and address the environmental concerns at the same time.

\begin{figure}[H]
	\centering
	\includegraphics[width=0.7\linewidth]{figures/total-ghg-2019-caption.jpg}
	\hfill
	\caption{Total U.S. GHG Emissions by Economic Sector in 2017 \cite{us_epa_sources_2020}.}
	\label{fig:ghg}
\end{figure}

% word on solar energy and the duck curve
To address these concerns, utilities rely more on intermittent renewable energy resources, wind and solar \cite{ming_resource_2019}.
However, high solar energy adoption creates a challenge. The need for electricity generators to quickly ramp up increases when the sun sets and the contribution from \gls{PV} falls \cite{us_department_of_energy_confronting_2017}.
The California ISO developed the “duck chart (or duck curve) depicted in Figure \ref{fig:duck} to
illustrate the difference between forecasted load and expected electricity production
from solar during a typical March day \cite{bouillon_prepared_2014}.

\begin{figure}[H]
	\centering
	\includegraphics[width=0.7\linewidth]{figures/caiso-duck.png}
	\hfill
	\caption{The duck curve \cite{bouillon_prepared_2014}.}
	\label{fig:duck}
\end{figure}

% solutions to the duck curve
The easiest solution to a demand ramp up is the increase of dispatchable generation, such as natural gas and coal \cite{bouillon_prepared_2014}.
Nonetheless, adding new gas and coal generation capacity is not consistent with reductions in carbon emissions.
Hence, our focus drifts to other potential low-carbon solutions, like nuclear generation and electricity storage \cite{ming_resource_2019}.

This article analyses the likelihood of the duck curve in the \gls{US} grid system. Then, it narrows down its focus on the \gls{UIUC}'s grid and studies some possible approaches to mitigate the duck curve effects. These approaches rely on nuclear energy, the production of hydrogen, and the use of hydrogen as an energy storage mechanism. 
% Link missing between this paragraph and the next section

%%%%%%%%%%%%%%%%%%%%%% TO DO
% Section \ref{section:hydroprod} briefly introduces two hydrogen production methods and their energy requirements. Section \ref{section:methodology}

\section{Nuclear Co-Generation}
\label{section:hydroprod}

This section presents the electrolysis process as a method to couple hydrogen production to a nuclear reactor.
Water electrolysis converts electric and thermal energy into chemical energy stored in hydrogen \cite{hi2h2_highly_2007}.
The process enthalpy change $\Delta H$ determines the required energy for the electrolysis reaction to take place.
Part of the energy corresponds to electric energy $\Delta G$ and the rest of it to thermal energy $T \cdot \Delta S$ \ref{fig:electro}.
In liquid water electrolysis or \gls{LTE}, electricity provides the thermal energy.
In steam electrolysis or \gls{HTE}, a high temperature heat source is necessary to supply the thermal energy.

\begin{figure}[H]
	\centering
	\includegraphics[width=0.7\linewidth]{figures/ele-heat_curve.png}
	\hfill
	\caption{Energy consumption of an ideal electrolysis process \cite{hi2h2_highly_2007}.}
	\label{fig:electro}
\end{figure}

In this analysis, the energy source (both electric and thermal) is a nuclear reactor with co-generation capabilities. The nuclear reactor produces electricity to supply the grid and electricity and thermal energy to produce hydrogen. \ref{fig:h2diag}

\begin{figure}[H]
	\centering
	\includegraphics[width=0.7\linewidth]{figures/hte-figure0.png}
	\hfill
	\caption{.}
	\label{fig:h2diag}
\end{figure}


\section{Methodology}
\label{method}



\section{Results}



\section{Conclusion}



\section{Acknowledgements}

Roberto E. Fairhurst Agosta and Prof. Huff are supported by the \gls{NRC} Faculty Development Program (award NRC-HQ-84-14-G-0054 Program B). Prof. Huff is also supported by the Blue Waters sustained-petascale computing project supported by the National Science Foundation (awards OCI-0725070 and ACI-1238993) and the state of Illinois, the DOE ARPA-E MEITNER Program (award DE-AR0000983), the DOE H2@Scale Program (award), and the International Institute for Carbon Neutral Energy Research (WPI-I2CNER), sponsored by the Japanese Ministry of Education, Culture, Sports, Science and Technology.

%%%%%%%%%%%%%%%%%%%%%%%%%%%%%%%%%%%%%%%%%%%%%%%%%%%%%%%%%%%%%%%%%%%%%%%%%%%%%%%%
\bibliographystyle{ans}
\bibliography{bibliography}
\end{document}
